\documentclass[10pt. letterpaper]{article}
%\documentclass[10pt. a4paper]{article}


%\usepackage{fullpage}
%\usepackage[margin=1in]{geometry}
%\usepackage[top=1in, bottom=1in, left=0.75in, right=0.75in, paperwidth=8.5in, paperheight=11in]{geometry}

\usepackage[margin=1in]{geometry}
\usepackage{amsmath, amssymb, amsfonts}

\usepackage{tikz,pgfplots}
\usepackage{graphicx}
\usepackage{float}


\def\eq1{y=\dfrac{x}{3x^2+x+1}}

\newcommand{\set}[1]{\setlength{\itemsep}{#1em}}


\newcommand\calculator{\tikz{
		\node (c) [inner sep=0pt, draw, fill=black, anchor=south west]{\phantom{N}};
		\begin{scope}[x=(c.south east),y=(c.north west)]    \fill[white] (.1,.7) rectangle (.9,.9);    
			\foreach \x in {.1, .33, .55, .79}{    
				\foreach \y in {.1, .24, .38, .53}{    
					\fill[white] (\x,\y) rectangle +(.11,.07);}} 
\end{scope} }}
\def\calcicon#1{\noindent#1 \calculator\ }



\begin{document}
	
\textbf{Critical Thinking Questions}


	\begin{figure}[H]
		\centering
		\includegraphics[width=0.4\textwidth]{limit}\\
		\caption{The Squeeze Theorem}
	\end{figure}

	
	\begin{enumerate}
	
	    \set{1.2}
		
		\item \calculator\ Let's examine the function $\eq1$.
		
		\item This is the symbol for set of all real numbers: $\mathbb{R}$	
		
		\item This is the symbol for set of all integers: $\mathbb{Z}$	
		
	    \item This is the symbol for set of all rationals: $\mathbb{Q}$	
	\end{enumerate}

	
\end{document}